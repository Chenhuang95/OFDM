%% LyX 2.3.4 created this file.  For more info, see http://www.lyx.org/.
%% Do not edit unless you really know what you are doing.
\documentclass[12pt,preprint]{elsarticle}
\usepackage[utf8]{inputenc}
\usepackage{amsmath}
\usepackage{amssymb}
\usepackage{graphicx}

\makeatletter

%%%%%%%%%%%%%%%%%%%%%%%%%%%%%% LyX specific LaTeX commands.
%% Because html converters don't know tabularnewline
\providecommand{\tabularnewline}{\\}

%%%%%%%%%%%%%%%%%%%%%%%%%%%%%% User specified LaTeX commands.
%%
%% Copyright 2007-2019 Elsevier Ltd
%%
%% This file is part of the 'Elsarticle Bundle'.
%% ---------------------------------------------
%%
%% It may be distributed under the conditions of the LaTeX Project Public
%% License, either version 1.2 of this license or (at your option) any
%% later version.  The latest version of this license is in
%%    http://www.latex-project.org/lppl.txt
%% and version 1.2 or later is part of all distributions of LaTeX
%% version 1999/12/01 or later.
%%
%% The list of all files belonging to the 'Elsarticle Bundle' is
%% given in the file `manifest.txt'.
%%

%% Template article for Elsevier's document class `elsarticle'
%% with numbered style bibliographic references
%% SP 2008/03/01
%%
%%
%%
%% $Id: elsarticle-template-num.tex 168 2019-02-25 07:15:41Z apu.v $
%%
%%


%% Use the option review to obtain double line spacing
%% \documentclass[authoryear,preprint,review,12pt]{elsarticle}

%% Use the options 1p,twocolumn; 3p; 3p,twocolumn; 5p; or 5p,twocolumn
%% for a journal layout:
%% \documentclass[final,1p,times]{elsarticle}
%% \documentclass[final,1p,times,twocolumn]{elsarticle}
%% \documentclass[final,3p,times]{elsarticle}
%% \documentclass[final,3p,times,twocolumn]{elsarticle}
%% \documentclass[final,5p,times]{elsarticle}
%% \documentclass[final,5p,times,twocolumn]{elsarticle}

%% For including figures, graphicx.sty has been loaded in
%% elsarticle.cls. If you prefer to use the old commands
%% please give \usepackage{epsfig}

%% The amssymb package provides various useful mathematical symbols
%% The amsthm package provides extended theorem environments
%% \usepackage{amsthm}

%% The lineno packages adds line numbers. Start line numbering with
%% \begin{linenumbers}, end it with \end{linenumbers}. Or switch it on
%% for the whole article with \linenumbers.
%% \usepackage{lineno}

\journal{Signal Processing}





\makeatother

\begin{document}
\begin{frontmatter}

%% Title, authors and addresses

%% use the tnoteref command within \title for footnotes;
%% use the tnotetext command for theassociated footnote;
%% use the fnref command within \author or \address for footnotes;
%% use the fntext command for theassociated footnote;
%% use the corref command within \author for corresponding author footnotes;
%% use the cortext command for theassociated footnote;
%% use the ead command for the email address,
%% and the form \ead[url] for the home page:
%% \title{Title\tnoteref{label1}}
%% \tnotetext[label1]{}
%% \author{Name\corref{cor1}\fnref{label2}}
%% \ead{email address}
%% \ead[url]{home page}
%% \fntext[label2]{}
%% \cortext[cor1]{}
%% \address{Address\fnref{label3}}
%% \fntext[label3]{}

\title{Three-Frequency Ranging and Positioning based on 5G OFDM Systems\tnoteref{t1,t2}}

\tnotetext[t1]{The work has been supported by the Joint Funds of the Ministry of
Education of China (No.6141A02022383).}

\tnotetext[t2]{The work has been supported by the Fundamental Research Funds for
the Central Universities of Ministry of Education of China (No.20101195611).}

%% use optional labels to link authors explicitly to addresses:
%% \author[label1,label2]{}
%% \address[label1]{}
%% \address[label2]{}

\author[1]{Wengang~Li\corref{cor1}\fnref{fn1}}

\ead{wgli@xidian.edu.cn}

\author[2]{Chen~Huang\fnref{fn2}}

\ead{huangchen@stu.xidian.edu.cn}

\author[3]{Liujiang Wang}

\ead{wanglj@stu.xidian.edu.cn}

\cortext[cor1]{Corresponding author}

\fntext[fn1]{Wengang Li is currently an associate professor with the School of
Communication Engineering, Xidian University.}

\fntext[fn2]{Chen Huang is currently pursuing the M.Sc. Degree with the School
of Communication Engineering, Xidian University.}

\address[1]{Xidian University, Xi’an, Shaanxi province, China}

\address[2]{Xidian University, Xi’an, Shaanxi province, China}

\address[3]{Xidian University, Xi’an, Shaanxi province, China}

%% Text of abstract
\begin{abstract}
Indoor positioning has great research value due to the problems of
indoor multipath environment and lack of GNSS signals. This paper
proposes a multi-frequency ranging and positioning system based on
OFDM communication system. With almost no impact on communication
capacity, three subcarriers of OFDM are used for ranging. The CRLB
of this ranging positioning system is proved. Ranging positioning
accuracy meets the requirements of indoor location applications. The
experimental simulation compares the performance with other indoor
positioning methods and proves the superiority of this system. This
paper selects specific subcarriers in the OFDM communication system
to be used for transmitting ranging frames and delay observations
without affecting other subcarriers used for communication. The theory
proves and simulates the relationship between ranging accuracy and
accuracy and channel parameters in a multipath environment. The simulation
results show that the positioning accuracy of about 5cm can be achieved
under the conditions of 5GHz frequency and high SNR.
\end{abstract}
%%Graphical abstract
\begin{graphicalabstract} %\includegraphics{grabs}
\end{graphicalabstract}

%%Research highlights
\begin{highlights}

\item 

This paper proposes a multi-frequency ranging and positioning system
based on OFDM communication system. With almost no impact on communication
capacity, three subcarriers of OFDM are used for ranging. Based on
OFDM position system, triple-frequency combining observation models
is proposed.

\item 

The CRLB of this ranging positioning system is proved. How to select
subcarrier frequency is proposed.\end{highlights}

%% keywords here, in the form: keyword \sep keyword

\begin{keyword}
OFDM\sep multi-frequency ranging\sep indoor location\sep CRLB %% PACS codes here, in the form: \PACS code \sep code

%% MSC codes here, in the form: \MSC code \sep code
%% or \MSC[2008] code \sep code (2000 is the default)
\end{keyword}
\end{frontmatter}

%% \linenumbers

%% main text

\section{Introduction}

With the development of society, location-based service (LBS) has
become more and more important. For example, robot positioning, vehicle
automatic driving, indoor rescue$^{\cite{Wang2012Handset}}$ , etc.
all require precise position information. Accurate position information
is the basis for developing LBS. At present, the increasingly mature
global navigation satellite system (GNSS) achieves sub-meter outdoor
positioning accuracy$^{\cite{Qin2019}}$, and the development of real-time
kinematic (RTK) technology improves the positioning accuracy to the
centimeter level with the help of ground-assisted stations$^{\cite{Brack2017Reliable}}$
. However, since satellite signals cannot penetrate buildings, indoor
positioning cannot be achieved with GNSS. Humans spend more than 70\%
of their time indoors, so the need for LBS for indoor activities is
more urgent. When GNSS cannot provide indoor positioning services,
how to obtain accurate indoor location information has become a research
hotspot. Industry and academia are seeking high-precision, high-reliability
indoor positioning technology, with a view to obtaining accurate location
information in electromagnetic environments and indoor environments
with complex geographical environments.

The indoor environment space is small, and the cellular network-based
partition positioning method$^{\cite{Xu2012Keeping}}$ cannot provide
sufficient positioning accuracy. The indoor radio frequency identification
(RFID) positioning method$^{\cite{Shangguan2016STPP}}$$^{\cite{Fu2017One}}$
has high accuracy, but the positioning distance is short, and special
equipment is needed. UWB, as a commonly used positioning technology,\citep{Liu2017Improved}
compared different positioning algorithms based on UWB ranging. \citep{Hu2018Relative}
used pseudo-random codes for indoor positioning in OFDM systems. \citep{GuiA}
Based on the fact that channel state information (CSI) can be used
for distance measurement, CSI-based CRLB is proposed in it, and the
theoretical research on positioning performance in a multipath environment
is carried out. The method of fingerprint characteristics in indoor
positioning technology is also a research hotspot. It uses machine
learning to locate the channel state information$^{\cite{Song2018CSI}}$,
Received Signal Strength Indicator(RSSI)$^{\cite{Zhang2015Wireless}}$$^{\cite{LiUHF}}$
and other characteristics. However, this method requires data collection
and network training in advance, which has greater limitations.

Wireless positioning technologies are mainly divided into two categories:
ranging-based technologies that measure propagation delay and non-ranging
technologies that use matching fingerprints, among which ranging-based
technologies are the first choice for lower cost. Because of its good
anti-multipath performance, Orthogonal Frequency Division Multiplexing
(OFDM) technology can be used to improve the accuracy of indoor ranging
and positioning. In\citep{WangOFDM}\citep{Wang2011Bounds} , the
Cramero Lower Bound (CRLB) based on OFDM was studied, and the single-path
channel was studied in \citep{WangOFDM}, and the performance of OFDM
was limited. In \citep{Wang2011Bounds} , the ranging of the multipath
channel is obtained in the frequency domain. Literature \citep{Wang70}
proved that with the same positioning accuracy, non-data-bearing orthogonal
frequency division multiplexing (NDB OFDM) in wireless positioning
is more energy-efficient than non-pulse-shaping pseudo-random noise
(NPS PN) currently widely used in satellites and the ground. Articles
\citep{DaiDistributed,LiRobust,Shahmansoori2017} improve positioning
accuracy by optimizing network and carrier power allocation.But the
signal power to locate the channel has higher requirements.Many positioning
algorithms are based on ranging information, such as the least squares
algorithm$^{\cite{NguyenLeast}}$ and the chan algorithm$^{\cite{MuUWB}}$.
Wang$^{\cite{Donglin2019Quasi}}$ proves the bound of position estimation
in 3-D localization is closely associated with the bound of range
estimation.That means the accuracy of distance measurement guarantees
the accuracy of positioning.

The communication and positioning integrated OFDM system proposed
in this paper can achieve good positioning accuracy while minimally
affecting the communication capacity, and can meet the needs of indoor
positioning applications. The three-frequency positioning method is
used to improve positioning accuracy and stability.

The contributions of this paper can be summarized as follows. 
\begin{enumerate}
\item Select three subcarriers in an OFDM communication system, and introduce
the selection method and the format of the ranging frame carried by
the subcarriers. 
\item we focus on OFDM systems and derive the bounds for OFDM ranging accuracy
in multipath channels based on Fisher information analysis. Prove
the EFI of multi-frequency ranging and how it is affected by the channel
parameters. 
\item Simulation experiments in this paper prove the results of theoretical
derivation. And compared with other indoor positioning methods ranging
positioning performance. It is verified how the ranging and positioning
accuracy changes with the change of channel parameters. 
\end{enumerate}

\section{OFDM position system}

OFDM technology is widely used in 4g, 5g and other high-speed communication
systems. The wireless local area network IEEE 802.11a standard, the
wireless metropolitan area network IEEE 802.16 standard, etc. all
adopt OFDM as the core technology. OFDM is a frequency division multiplexing
technique based on multi-carrier modulation. Multiple subcarriers
overlap each other orthogonally in frequency to form multiple subchannels,
as shown in FIG. 1. Through serial-to-parallel conversion, the input
high-rate data information stream is converted into multiple low-rate
data information streams that are transmitted in parallel, while the
broadband is converted into multiple identical narrowbands, and the
parallel data streams are transmitted on narrowband subchannels. Under
normal communication conditions, the signal bandwidth of an OFDM system
is less than the relevant bandwidth of the channel. Therefore, the
fading of each subchannel is consistent with flat fading, so it has
good resistance to multipath.

\subsection{OFDM modulation and demodulation}

\begin{figure}[htbp]
\begin{centering}
\includegraphics[width=3.5in]{\string"picture/OFDM transmitter\string".eps} 
\par\end{centering}
\caption{OFDM modulation process}
\end{figure}

OFDM modulation method As shown in Figure 1, Serial to Parallel Conventor
converts a high-speed data stream into N sets of parallel data blocks
$D_{0}$ \textasciitilde{} $D_{N-1}$ , and modifies each data block
with a subcarrier $f_{0}$ \textasciitilde$f_{N-1}$ . The modulated
subcarrier signals of the N parallel branches are added to obtain
the actual transmitted signal of the OFDM.

\begin{figure}[htbp]
\begin{centering}
\includegraphics[width=3.5in]{\string"picture/OFDM receiver\string".eps} 
\par\end{centering}
\caption{Receiver OFDM demodulation process}
\end{figure}

As shown in FIG. 2, at the receiving end, the received signal enters
N parallel branches at the same time, and can be restored by multiplying
and integrating (coherent demodulation) with N subcarriers, respectively.
Parallel data is merged into serial data through Parallel-to-Serial
Conventor.

\subsection{multi-subcarrier ranging positioning}

The OFDM ranging method proposed in this paper uses three subcarriers
$f_{0}$, $f_{1}$, $f_{2}$which are specifically used for ranging,
and the remaining subcarriers are used to transmit data just like
the communication system. As shown in Figure 3, suppose there are
N subcarriers in an OFDM system, and $f_{0}$ is the subcarrier with
the smallest baseband frequency.The subcarrier frequency is determined
in advance during positioning, and the three subcarrier data are skipped
during parallel-to-serial conversion at the receiver for ranging alone.

\begin{figure}[htbp]
\begin{centering}
\includegraphics[width=3.5in]{picture/subCarriers} 
\par\end{centering}
\caption{OFDM power spectrum diagram}
\end{figure}

Data such as the coordinates of the base station are transmitted on
the ranging subcarriers. The data frame format is shown in Figure
4.

The ranging frame structure is shown in Figure 4. The first two bytes
of the frame. Check sum One byte, used to check the data. The base
station's latitude, longitude and altitude are 8 bytes each, for a
total of 24 bytes. The device id is one byte, which indicates the
signal transmitted by which base station. The subcarrier id is one
byte and is used to indicate which subcarrier signal. The ranging
signal delay is used to measure the distance between the receiver
and the base station. The three subcarriers used for ranging can not
only improve the ranging accuracy but also ensure the accuracy of
the coordinate data of the ranging base station. Occupying three subcarriers
at the same time will not cause much impact on the communication capacity
of the original communication system.
\begin{figure}[htbp]
\begin{centering}
\textsf{\includegraphics[width=7in]{\string"picture/position frame\string".eps}}
\par\end{centering}
\caption{Ranging Data Frame}
\end{figure}


\subsection{channel fading}

Due to the subcarriers of OFDM system are orthogonal to each other,
they are independent of each other when transmitting in the channel,
and the attenuation in the channel is different. Combining observation
of multiple subcarriers can improve the stability and accuracy of
ranging and positioning. Under normal communication conditions, the
sub-signal bandwidth of orthogonal frequency division multiplexing
system is smaller than the relevant bandwidth of the channel, therefore,
the fading of each sub-channel can be regarded as consistent with
flat fading, so it has good multipath resistance. When an OFDM signal
passes through a wireless multi-path channel, frequency selective
fading causes several groups of subcarriers in the OFDM symbol to
have larger fading. The large fading in the frequency response of
this channel will distort the information carried on adjacent subcarriers.
Therefore, maximizing the frequency difference when selecting subcarriers
for positioning can resist frequency selective fading. In frequency
selective fading, the coherence bandwidth of the channel is smaller
than the bandwidth of the signal. Therefore, different frequency components
of the signal experience uncorrelated fading. At this time, the use
of multiple subcarriers can improve the stability of observation.
When one subcarrier signal produces large fading, other subcarriers
are less affected. Combined observation can ensure the accuracy of
observation results.

\section{Triple-frequency combining observation models}

As there is a certain correlation between the errors of the observation
values with different frequencies, the combination observation values
formed by linear combination of multiple observation values can achieve
the purpose of weakening various errors and improving positioning
accuracy. Based on OFDM signals and multipath channels, a three-frequency
observation model is proposed to eliminate the influence of noise.

\subsection{signal and channel model}

The transmission ranging signal of the OFDM system is expressed as

\begin{equation}
\begin{aligned}s_{0}(t) & =s(t)g_{0}(t)\\
 & =\sum_{k=0}^{N-1}A_{k}\cos\left(2\pi\left(f_{\mathrm{c}}+f_{k}\right)t+\phi_{k}\right)g_{0}(t)
\end{aligned}
\end{equation}
Where N is the number of subcarriers, ${f_{\text{c}}}$ is the central
frequency, ${f_{k}}$ is the baseband frequency of the th subcarrier
and can be expressed as ${f_{k}}=(2k-N+1)\Delta f/2$, ${{\text{A}}_{k}}$
and ${\phi_{k}}$ are the amplitude and phase of the designed symbol
for ranging modulated on the kth subcarrier. The unit rectangular
pulse ${g_{0}}(t)$ is non-zero for $t\in\left[{-{T_{{\text{CP}}}},T}\right]$
where $T=1/\Delta f$ and ${T_{{\text{CP}}}}$ is the length of the
cyclic prefix. $\Delta f$ is the subcarrier spacing of ranging signal.

Due to the reflection of signals from indoor walls, there are multipath
signals in the environment, so the signal received by the receiver
can be expressed as: 
\begin{equation}
x(t)\triangleq\left[{h(t)*{s_{0}}(t)}\right]g(t)
\end{equation}
$h(t)$ is the impulse response of the multipath channel and can be
written as 
\begin{equation}
h(t)=\sum\limits _{l=0}^{L-1}{a_{l}}\delta\left({t-{\tau_{l}}}\right)
\end{equation}
Where $n(t)$ is the additive white Gaussian noise with two-sided
spectrum density of ${{\text{N}}_{\text{0}}}/2$, L is the number
of arrival paths, ${a_{l}}$ and ${\tau_{l}}$ are the amplitude and
propagation delay of the pth path, $\ensuremath{L}$ is the number
of arrival paths, $\ensuremath{{\tau_{{\rm {0}}}}}$is the propagation
delay of the first path and can be expressed as ${\tau_{0}}=R/c$,
where $R$ is the distance between the transmitter and the base station,
$c$ is the speed of light. Therefore, the estimation of the distance
can be converted into the estimation of the delay. The unknown parameters
to be estimated can be organized as ${\bf {\theta}}={\left[{{{\bf {\tau}}^{T}},{{\bf {\alpha}}^{T}}}\right]^{T}}$,
where ${\mathbf{\tau}}={\left[{{\tau_{0}},{\tau_{1}},\ldots,{\tau_{L-1}}}\right]^{T}}$
and $\alpha{\rm {=}}{\left[{{\alpha_{0}},{\alpha_{1}},\ldots,{\alpha_{L-1}}}\right]^{T}}$.The
received signal without the cyclic prefix is expressed as

\begin{equation}
r(t)=x(t)+n(t){\text{ }}g(t)
\end{equation}
Where $g(t)$ is a unit rectangular pulse which is non-zero in {[}0,T{]}.The
Fourier Transform of $r(t)$ can be expressed as ${\text{R(f) = X(f) + N(f)}}$.
Where ${\text{X(f)}}$ is the Fourier Transform of ${\text{x(t)}}$
given by

\begin{equation}
\begin{aligned}X(f)= & \sum_{k=0}^{N-1}\frac{A_{k}}{2}\left[H\left(f_{\mathrm{c}}+f_{k}\right)G\left(f-f_{\mathrm{c}}-f_{k}\right)e^{j\phi_{k}}\right.\\
 & \left.+H\left(-f_{\mathrm{c}}-f_{k}\right)G\left(f+f_{\mathrm{c}}+f_{k}\right)e^{-j\phi_{k}}\right]
\end{aligned}
\end{equation}
The Fourier Transform of $r(t)$ isexpressed as

\begin{equation}
\ensuremath{\begin{array}{c}
R(f)=R\left(f\right)=X\left(f\right)+N\left(f\right)\\
=\frac{{T{A_{k}}}}{2}\sum\limits _{l=1}^{L}{\alpha_{l}}{e^{-j\left({2\pi f{\tau_{l}}-{\phi_{k}}}\right)}}+N(f)
\end{array}}
\end{equation}
and G(f), H(f),and N(f) are the Fourier Transforms of g(t), h(t),and
n(t),respectively.

\subsection{combining observation models}

In the process of positioning and ranging, the delay time of the signal
actually corresponds to the change of the signal carrier phase. As
shown in figure 4.6. The phase of the ranging signal when it is emitted
from the transmitting end is ${\phi_{S}}$, and the phase when it
is received by the receiver is ${\phi_{E}}$. The phase difference
of the signals can be expressed as:

\begin{equation}
\varphi={\phi_{E}}-{\phi_{S}}
\end{equation}

The distance between receiver and transmitter is expressed as:

\begin{equation}
\rho=c\tau=\varphi\lambda
\end{equation}
where $\rho$ is the distance, $c$ is the speed of light, $\tau$
is the signal delay, and $\lambda$ is the carrier wavelength. For
a 5GHz carrier wave, one wavelength is 0.06m, so the ambiguity of
the receiver observation will lead to ranging errors of more than
one wavelength. Therefore, multi-subcarrier observation model is used
to improve the accuracy of receiver observation.

\begin{figure}[htbp]
\begin{centering}
\textsf{\includegraphics[width=6in]{\string"picture/carrier ranging\string".eps}}
\par\end{centering}
\caption{Carrier ranging process}
\end{figure}

The frequencies of the ranging subcarriers are determined in advance
to be${f_{{\rm {0}}}}$, ${f_{1}}$, and ${f_{2}}$. Reach the receiver
after propagating through the wireless channel. The observation of
the ranging subcarrier phase measured by the receiver can be expressed
as:

\begin{equation}
{\varphi_{i}}=\frac{1}{{\lambda_{i}}}\rho-{N_{i}}+{\varepsilon_{i}}
\end{equation}
In Eq. (7), $i={\rm {1}},{\rm {2}},{\rm {3}}$ represents different
carrier frequencies,${\varphi_{i}}$ represents the carrier phase
observation value, $\rho$ represents the true value of the geometric
distance from the base station transmitter to the receiver, and ${{\cal E}_{i}}$
is the carrier phase noise of the corresponding frequency. ${N_{i}}$
is the ambiguity of the observed value. The observation values of
the OFDM subcarrier phases of different frequencies represented by
Eq. (1) are linearly combined, and the observation values of the combined
subcarrier phases are expressed as:

\begin{equation}
\begin{aligned}\varphi_{ijk}=i\varphi_{1}+j\varphi_{2}+k\varphi_{3} & =\left(\frac{i}{\lambda_{1}}+\frac{j}{\lambda_{2}}+\frac{k}{\lambda_{3}}\right)\rho\\
 & -\left(iN_{1}+jN_{2}+kN_{3}\right)\\
 & +\left(i\varepsilon_{1}+j\varepsilon_{2}+k\varepsilon_{3}\right)
\end{aligned}
\end{equation}
In Eq. (8), ${\varphi_{ijk}}$ is the carrier phase combined observation
value, and $(i,j,k)$ is the combined observation value coefficient.
In order to ensure the whole cycle characteristic of the combined
ambiguity of the combined observation value,which can be writtened
as ${N_{ijk}}=i{N_{1}}+j{N_{2}}+k{N_{3}}$. $i,j,k$ are non-zero
integers, and 
\begin{equation}
\frac{1}{{\lambda_{ijk}}}=\frac{i}{{\lambda_{1}}}+\frac{j}{{\lambda_{2}}}+\frac{k}{{\lambda_{3}}}
\end{equation}
the wavelength F of the combined phase combination observations is
solved as: 
\begin{equation}
{\lambda_{ijk}}=\frac{{{\lambda_{1}}{\lambda_{2}}{\lambda_{3}}}}{{i{\lambda_{2}}{\lambda_{3}}+j{\lambda_{1}}{\lambda_{3}}+k{\lambda_{1}}{\lambda_{2}}}}
\end{equation}
then Eq. (8) can be simplified as:

\begin{equation}
{\varphi_{ijk}}=\frac{1}{{\lambda_{ijk}}}\rho-{N_{ijk}}+{\varepsilon_{ijk}}
\end{equation}
where ${N_{ijk}}$ is the ambiguity of the combined observations and
${\varepsilon_{ijk}}=i{\varepsilon_{1}}+j{\varepsilon_{2}}+k{\varepsilon_{3}}$
is the noise of the combined observations.

Due to the linear combination, the noise amplification factor of the
combined observation value will be greater than the noise amplification
factor corresponding to the original carrier observation value. The
formula of the noise amplification factor in units of length is rewritten
into the frequency form:

\begin{equation}
{\bar{n}_{ijk}}=\frac{{\sqrt{{i^{2}}+{j^{2}}+{k^{2}}}}}{{i{f_{1}}+j{f_{2}}+k{f_{3}}}}\to\min
\end{equation}
In the coefficient space $\left(i\text{,}j,k\right)$ the combination
that satisfies the above formula is a straight line.

\begin{equation}
\left[{\begin{array}{l}
i\\
j\\
k
\end{array}}\right]=t\left[{\begin{array}{l}
{f_{i}}\\
{f_{j}}\\
{f_{k}}
\end{array}}\right]
\end{equation}
It can be known from Eq. (15) that the noise amplification factor
in terms of length is the smallest when the combination coefficient
is proportional to the subcarrier frequency.

Based on Eq.(13), we can obtain the distance from the phase of the
signal. 
\begin{equation}
\ensuremath{\rho=\left({{N_{ijk}}+{\varphi_{ijk}}+{\varepsilon_{ijk}}}\right){\lambda_{ijk}}=c{\tau_{0}}}
\end{equation}

Therefore, in the OFDM ranging system, the combination coefficient
is determined according to the predetermined subcarrier frequency
to ensure that the result is affected by noise Minimal, it is helpful
to improve the accuracy of the ambiguity solution. The geometric distance
between the transmitting end and the receiving end is calculated by
combining the phase observations.

\section{positioning algorithm and technology}

Positioning accuracy is closely related to the accuracy of ranging
information\citep{Donglin2019Quasi}. Algorithms that use ranging
information to calculate the receiver's three-dimensional coordinates
usually require signals from four transmitters to locate. The receiver
measures the distance from the four base stations and knows the coordinates
of the base stations. The coordinates of the receiver are obtained
through geometric positioning algorithm.Fig. 6 illustrates the indoor
positioning method. The receiver measures the distances $R_{1}$,
$R_{2}$, $R_{3}$, $R_{4}$ and the coordinates of the base stations
from the four base stations, and obtains the coordinates of the receiver
through a geometric positioning algorithm. Geometric positioning methods,
such as the least squares method, are closely related to positioning
accuracy.

\begin{figure}[htbp]
\begin{centering}
\textsf{\includegraphics[width=3.5in]{\string"picture/four anchors position\string".eps}}
\par\end{centering}
\caption{Indoor wireless positioning}
\end{figure}

Since OFDM usually uses antenna arrays, multiple antennas can be used
for positioning in one transmitter, which can reduce the number of
transmitters required for positioning compared with common methods,
as shown in Fig. 7.The positioning algorithm for multiple antennas
and multiple base stations is the same, except that the position coordinates
of the antennas are substituted for the position coordinates of the
base stations.

\begin{figure}[htbp]
\begin{centering}
\textsf{\includegraphics[width=3.5in]{\string"picture/antenna position\string".eps}} 
\par\end{centering}
\caption{multi-antenna wireless positioning}
\end{figure}

The next chapter proves the ranging accuracy of this system.The algorithm
can be summarized as follows. 
\begin{enumerate}
\item Determine the subcarrier frequency used for ranging. 
\item The receiver broadcasts the request positioning signal and prepares
to receive it. 
\item The receiver multiple receives positioning signals from transmitters. 
\item Extracting subcarrier signals and combine subcarrier observations. 
\item Get the distances between the receiver and the transmitters. 
\item Calculate receiver coordinates using geometric algorithms. mathematical
model of base station positioning 
\end{enumerate}
As shown in the figure 6, the four anchors are deployed in advance
as base stations, and the position coordinates of the four anchors
are calibrated and recorded as $A0\left(x_{a0},y_{a0},z_{a0}\right)$,$A1\left(x_{a1},y_{a1},z_{a1}\right)$,$A2\left(x_{a2},y_{a2},z_{a2}\right)$,
and $A3\left(x_{a3},y_{a3},z_{a3}\right)$, respectively. At a certain
time k, the distances between the receiver and the four antennas are
$R_{1},R_{2},R_{3},R_{4}$, respectively.

The base station positioning principle is to measure the distance
between a receiver and 3 or more base stations , and solve the distance
equation to get the position coordinates of the measured receiver,
which can be expressed as follows: 
\begin{equation}
R\left(x,y,z\right)=\sqrt{\left(x-x_{a}\right)^{2}+\left(y-y_{a}\right)^{2}+\left(z-z_{a}\right)^{2}}
\end{equation}

Among them,$\left(x,y,z\right)$ is the receiver coordinate to be
solved, $\left(x_{a},y_{a},z_{a}\right)$ is the base station coordinate
calibrated in advance, and R is the geometric distance between the
receiver and base stations.

\subsection{ranging positioning analysis}

When using our proposed positioning method, i.e. four base stations
to locate a target point, the following four situations generally
occur.

Case 1: Each observation is within the normal error range, which is
in line with the statistical characteristics of the measurement error.

In this case, the four ranging circles may not converge at one point,
but form a converging area with a relatively small area. The ideal
situation is that the four circles meet at one point. This situation
only occurs when all positioning base station systems have no errors.

Case 2: One of the base stations has a large ranging error, and the
other base stations have normal ranging errors.

This situation usually occurs in an environment where a certain signal
propagation path is blocked. If in this case the solution of the anchor
point equations directly will lead to an increase in the uncertainty
of the results. In order to get better positioning results, the best
way is to find the abnormal ranging device first and then remove it
from the solving equations; the next best method is to weight the
processing to reduce the impact on the positioning results.

Case 3: All ranging circles do not form an intersection area.

This situation of no intersection area is caused by the signals of
the four ranging devices being blocked. It does not have a least squares
solution. At this time, the least squares cannot converge, and the
correct positioning point cannot be found.

Case 4: Only two base stations have ranging information.

This situation occurs when two base stations are able to perform normal
ranging normally, and the other two base stations are unable to operate
normally and cannot perform normal ranging. In this case, there are
only two ranging information. In three-dimensional space, only two
ranging information cannot be solved. If approximate positioning points
are required, additional constraint information needs to be added.

\subsection{nonlinear least squares iteration}

According to the ranging information of the four base stations, the
corresponding ranging equations can be obtained, and the coordinates
of the target positioning can be obtained by solving the ranging equations.
In this article, we use nonlinear least squares iteration to solve
this ranging equations.

Assuming that the estimated value of the TAG coordinate is $\hat{\boldsymbol{\beta}}=\left(\hat{x},\hat{y},\hat{z}\right)$,
the first-order Taylor expansion of the ranging Eq.(17) at $\hat{\boldsymbol{\beta}}$
is obtained as:

\begin{equation}
R\left(x,y,z\right)=R\left(\hat{x},\hat{y},\hat{z}\right)+\frac{\partial R}{\partial x}\cdot\left(x-\hat{x}\right)+\frac{\partial R}{\partial y}\cdot\left(y-\hat{y}\right)+\frac{\partial R}{\partial z}\cdot\left(z-\hat{z}\right)+\mathcal{O}\left(\hat{\boldsymbol{\beta}}\right)^{2}
\end{equation}

In order to facilitate expression and perform matrix operations, the
following equation holds:

\begin{equation}
\triangle R=R\left(x,y,z\right)-R\left(\hat{x},\hat{y},\hat{z}\right)=R-\hat{R}
\end{equation}

\begin{equation}
\triangle x=x-\hat{x},\triangle y=y-\hat{y},\triangle z=z-\hat{z}
\end{equation}

\begin{equation}
\hat{r}=\sqrt{\left(\hat{x}-x_{a}\right)^{2}+\left(\hat{y}-y_{a}\right)^{2}+\left(\hat{z}-z_{a}\right)^{2}}
\end{equation}

According to Eqs.(17-21) and the partial differential formula, the
following equations can be obtained:

\begin{equation}
\frac{\partial R}{\partial x}=\frac{x-\hat{x}}{\hat{r}},\frac{\partial R}{\partial y}=\frac{y-\hat{y}}{\hat{r}},\frac{\partial R}{\partial z}=\frac{z-\hat{z}}{\hat{r}}
\end{equation}

To facilitate matrix operations, let the following equation be:

\begin{equation}
\boldsymbol{u}=\left[\frac{\partial R}{\partial x},\frac{\partial R}{\partial y},\frac{\partial R}{\partial z}\right]
\end{equation}

If there are m sets of ranging information, and the second order and
above errors of Eq.(17) are discarded, the nonlinear ranging equation
can be linearized into the following equations:

\begin{equation}
\left[\begin{array}{c}
\triangle R_{1}\\
\triangle R_{2}\\
\vdots\\
\triangle R_{m}
\end{array}\right]=\left[\begin{array}{ccc}
\frac{\partial R_{1}}{\partial x} & \frac{\partial R_{1}}{\partial y} & \frac{\partial R_{1}}{\partial z}\\
\frac{\partial R_{2}}{\partial x} & \frac{\partial R_{2}}{\partial y} & \frac{\partial R_{2}}{\partial z}\\
\vdots & \vdots & \vdots\\
\frac{\partial R_{m}}{\partial x} & \frac{\partial R_{m}}{\partial y} & \frac{\partial R_{m}}{\partial z}
\end{array}\right]\cdot\left[\begin{array}{c}
\triangle x\\
\triangle y\\
\triangle z
\end{array}\right]
\end{equation}

Write Eq.(24) as a matrix, as follows:

\begin{equation}
\mathbf{\triangle\boldsymbol{R}=H\cdot\triangle\hat{\boldsymbol{\beta}}}
\end{equation}

Where,

\begin{equation}
\triangle\boldsymbol{R}=\left[\begin{array}{c}
\triangle R_{1}\\
\triangle R_{2}\\
\vdots\\
\triangle R_{m}
\end{array}\right],\mathbf{H}=\left[\begin{array}{ccc}
\frac{\partial R_{1}}{\partial x} & \frac{\partial R_{1}}{\partial y} & \frac{\partial R_{1}}{\partial z}\\
\frac{\partial R_{2}}{\partial x} & \frac{\partial R_{2}}{\partial y} & \frac{\partial R_{2}}{\partial z}\\
\vdots & \vdots & \vdots\\
\frac{\partial R_{m}}{\partial x} & \frac{\partial R_{m}}{\partial y} & \frac{\partial R_{m}}{\partial z}
\end{array}\right],\triangle\hat{\boldsymbol{\beta}}=\left[\begin{array}{c}
\triangle x\\
\triangle y\\
\triangle z
\end{array}\right]
\end{equation}

Finally, the least square solution of Eq.(25) is obtained as follows:

\begin{equation}
\triangle\hat{\boldsymbol{\beta}}=\left(\mathbf{H}^{T}\cdot\mathbf{H}\right)^{-1}\cdot\mathbf{H}^{T}\cdot\triangle\boldsymbol{R}
\end{equation}

$\mathbf{H}$ is the observation matrix, which is a Jacobian matrix.
From the above Eq.(18) to (27), the solution process of the nonlinear
least squares is given.

The specific solution steps are given below:

Step 1:Sets the initial value of the nonlinear least squares iteration.

Set to $\hat{\boldsymbol{\beta}}=\left[0,0,0\right]^{T}$ or the estimated
value at the previous moment, i.e. $\hat{\boldsymbol{\beta}}=\hat{\boldsymbol{\beta}}_{k-1}$.

Step 2: Calculate the prior ranging error $\triangle\boldsymbol{R}$
by using Eq.(19).

Step 3: Calculate the cosine vector $\mathbf{u}$ and the observation
matrix $\mathbf{H}$ of the ranging direction by using Eqs.(21),(22).

Step 4: Use Eq.(27) to solve the state estimation error vector $\triangle\hat{\boldsymbol{\beta}}$.

Step 5: according to Eq.(20), update the current iteration state:
$\hat{\boldsymbol{\beta}}=\hat{\boldsymbol{\beta}}+$$\triangle\hat{\boldsymbol{\beta}}$.

Step 6: Judge whether convergence. The convergence condition is: $\left\Vert \triangle\hat{\boldsymbol{\beta}}\right\Vert _{2}\leq Th$,
generally $Th=10^{-6}$.

If the convergence condition is satisfied, then $\hat{\boldsymbol{\beta}}$
at this time is the state least squares estimation value of the current
epoch, and step 7 is continued. Otherwise, return to step 2 and continue
the iteration. A number of iterations can be set to prevent a dead
loop from always failing to meet the convergence conditions. Generally,
convergence can be achieved in 3-5 iterations.

Step 7: Calculate the estimation error.

\section{Analysis of Ranging Accuracy}

Cramer-Rao Lower Bound (CRLB) can be used to calculate the best estimation
accuracy that can be obtained in unbiased estimation, so it is often
used to calculate the best estimation accuracy that can be achieved
by theory. The simplest form of CRLB is the reciprocal of Fisher information.
This section analyzes the ranging accuracy by calculating CRLB. Based
on the signal and channel models in chapter 3, the likelihood function
of $\theta$ satisfies the following conditions

\begin{equation}
\begin{aligned}\Lambda(R(f) & ;\boldsymbol{\theta})\\
 & \propto\exp\left(-\frac{1}{N_{\mathrm{f}}}\left\Vert R(f)-\frac{TA_{k}}{2}\sum_{l=1}^{L}\alpha_{l}e^{-j\left(2\pi f\tau_{l}-\phi_{k}\right)}\right\Vert ^{2}\right)
\end{aligned}
\end{equation}
where ${\bf {\theta}}={\left[{{{\bf {\tau}}^{T}},{{\bf {\alpha}}^{T}}}\right]^{T}}$,
$\boldsymbol{\theta}$ is a matrix of signal delay and signal amplitude.
Since the noise components at different samples in $\mathcal{R}_{\mathrm{f}}$
are i.i.d. complex Gaussian, $\Lambda(\cdot;\boldsymbol{\theta})$
can be expressed as the product of the likelihood functions at different
sampled frequencies.According to the log-likelihood function, the
Score function is calculated. Calculate the second-order moment of
the Score function. Fisher information can be obtained

\begin{equation}
{{\mathbf{F}}_{\mathbf{\theta}}}=\left[{\begin{array}{lc}
{{\mathbf{F}}_{{\mathbf{\tau\tau}}}} & {{\mathbf{F}}_{{\mathbf{\tau a}}}}\\
{{\mathbf{F}}_{{\mathbf{a\tau}}}} & {{\mathbf{F}}_{{\mathbf{aa}}}}
\end{array}}\right]
\end{equation}
where

\begin{equation}
\mathbf{F}_{\theta}=\mathbb{E}_{\mathcal{R}_{\mathrm{f}}}\left\{ -\frac{\partial^{2}\ln\Lambda\left(\mathcal{R}_{\mathrm{f}};\boldsymbol{\theta}\right)}{\partial\boldsymbol{\theta}\partial\boldsymbol{\theta}^{\mathrm{T}}}\right\} 
\end{equation}
It can be known from the above formula that the Fisher information
of the positioning parameter is independent of the symbol phase.The
mean squared error (MSE) of ${\mathbf{\theta}}$ is bounded as

\begin{equation}
\mathbb{E}_{\mathcal{R}_{\mathrm{f}}}\left\{ (\hat{\boldsymbol{\theta}}-\boldsymbol{\theta})(\hat{\boldsymbol{\theta}}-\boldsymbol{\theta})^{\mathrm{T}}\right\} \succeq\mathbf{F}_{\boldsymbol{\theta}}^{-1}
\end{equation}
$\mathbf{F}_{\boldsymbol{\theta}}$ can be calculated as the sum of
the FIMs for all subcarriers as shown in Eq. (32).

\begin{equation}
\mathbf{F}_{\theta}=\sum_{k=0}^{N-1}\left(\mathbf{F}\left(f_{\mathrm{c}}+f_{k}\right)+\mathbf{F}\left(-f_{\mathrm{c}}-f_{k}\right)\right)
\end{equation}
the FIM at frequency f $\mathbf{F}(f)$ can be calculated from Eqs.
(28), (30)

\begin{equation}
\mathbf{F}(f)=\mathbb{E}_{\mathcal{R}_{\mathrm{f}}}\left\{ -\frac{\partial^{2}\ln\Lambda(R(f);\boldsymbol{\theta})}{\partial\boldsymbol{\theta}\partial\boldsymbol{\theta}^{\mathrm{T}}}\right\} 
\end{equation}
After somealgebra, theelements in $\mathbf{F}(f)$ are calculated
as follows,

\begin{equation}
\mathbb{E}_{\mathcal{R}_{\mathrm{f}}}\left\{ -\frac{\partial^{2}\ln\Lambda(R(f);\boldsymbol{\theta})}{\partial\tau_{i}\partial\tau_{j}}\right\} =\frac{\alpha_{i}\alpha_{j}T}{N_{0}}P_{k}(2\pi f)^{2}\cos\left(2\pi f\tau_{ij}\right)
\end{equation}

\begin{equation}
\mathbb{E}_{\mathcal{R}_{\mathrm{f}}}\left\{ -\frac{\partial^{2}\ln\Lambda(R(f);\boldsymbol{\theta})}{\partial\tau_{i}\partial\alpha_{j}}\right\} =-\frac{\alpha_{i}T}{N_{0}}P_{k}(2\pi f)\sin\left(2\pi f\tau_{ij}\right)
\end{equation}

\begin{equation}
\mathbb{E}_{\mathcal{R}_{\mathrm{f}}}\left\{ -\frac{\partial^{2}\ln\Lambda(R(f);\boldsymbol{\theta})}{\partial\alpha_{i}\partial\alpha_{j}}\right\} =\frac{T}{N_{0}}P_{k}\cos\left(2\pi f\tau_{ij}\right)
\end{equation}
where $\tau_{ij}=\tau_{i}-\tau_{j}$ , $0\leq i,j\leq L-1$ and $P_{k}=A_{k}^{2}$
is the power of the $k$th subcarrier. As can be seen from Eqs. (34),
(35) and (36), FIM is independent of the phase of the data. Substitute
the above formula into Eq. (20), we obtain the expressions of the
elements in $\mathbf{F}_{\boldsymbol{\theta}}$. The CRLB for the
mean-square estimation error of $\tau_{0}$ is $\sigma_{\mathrm{CRB}}^{2}\left(\tau_{0}\right)$.
The CRB can be written as

\begin{equation}
\mathbb{E}_{\mathcal{R}_{f}}\left[\left(\hat{\tau}_{0}-\tau_{0}\right)^{2}\right]\geq\sigma_{\mathrm{CRB}}^{2}\left(\tau_{0}\right)=\left[\mathbf{F}_{\theta}^{-1}\right]_{1,1}
\end{equation}
where $\left[\mathbf{F}_{\theta}\right]_{1,1}$ is the equivalent
Fisher information (EFI) of $\tau_{0}$. Since $f_{\mathrm{c}}$ is
much larger than the bandwidth for most OFDM signals, we approximate
the elements in $\mathbf{F}_{\boldsymbol{\theta}}$ except for $\left[\mathbf{F}_{\theta}\right]_{1,1}$
as follows

\begin{equation}
\sum P_{k}\omega_{\mathrm{c}k}^{2}\cos\left(\omega_{\mathrm{c}k}\tau_{ij}\right)\approx\omega_{\mathrm{c}}^{2}\sum P_{k}\cos\left(\omega_{\mathrm{c}k}\tau_{ij}\right)
\end{equation}

\begin{equation}
\sum P_{k}\omega_{\mathrm{c}k}\sin\left(\omega_{\mathrm{c}k}\tau_{ij}\right)\approx\omega_{\mathrm{c}}\sum P_{k}\sin\left(\omega_{\mathrm{c}k}\tau_{ij}\right)
\end{equation}

\begin{equation}
\sum P_{k}\omega_{\mathrm{c}k}^{2}\approx2\omega_{\mathrm{c}}^{2}E_{\mathrm{T}}/T
\end{equation}

In multipath channels, the performance of ranging is de- graded due
to the overlap between the first path and the subsequent paths in
the received waveform$^{\cite{Shen2010Fundamental}}$. Since the second
path is closest to the first path, in the following analysis, we focus
mainly on the performance analysis for ranging in the two-path channel.
The approximate EFI for ranging based on Eqs. (38), (39), and (40)
in the two-path channel is given by

\begin{equation}
\left[\mathbf{F}_{\theta}\right]_{1,1}=\frac{2\alpha_{0}^{2}T}{N_{0}}\left(\sum_{k=0}^{N-1}P_{k}\omega_{\mathrm{c}k}^{2}-\frac{\omega_{\mathrm{c}}^{2}T}{2E_{\mathrm{T}}}h\right)\label{Eq.(39)}
\end{equation}
where 
\begin{equation}
h=\left(\sum_{k=0}^{N-1}P_{k}\cos\left(\omega_{k}\tau\right)\right)^{2}+\left(\sum_{k=0}^{N-1}P_{k}\sin\left(\omega_{k}\tau\right)\right)^{2}\label{Eq.(40)}
\end{equation}
and $\tau=\tau_{1}-\tau_{0}$. The first term of the polynomial in
\ref{Eq.(39)} is the EFI of ranging in the single- path channel,
and the second term represents the effect of path-overlap in multipath
channels.we will quantify the impact of OFDM channel parameters on
EFI based on \ref{Eq.(40)}.We consider total energy constraint, in
which the energy of the FFT part of one OFDM symbol is constrained
as 
\begin{equation}
\int_{0}^{T}{s^{2}}(t)dt=\frac{T}{2}\sum\limits _{k=0}^{N-1}{P_{k}}\leqslant{E_{\text{T}}}
\end{equation}
where ${E_{\text{T}}}$ is the maximum total transmit energy.Based
on \ref{Eq.(39)} and \ref{Eq.(40)}, EFI can be rewritten as

\begin{equation}
\begin{aligned}\left[\mathbf{F}_{\theta}\right]_{1,1}= & \frac{2\alpha_{0}^{2}T}{N_{0}}\left[\left(\frac{2E_{\mathrm{T}}}{T}-\frac{Th}{2E_{\mathrm{T}}}\right)\omega_{\mathrm{c}}^{2}\right.\\
 & \left.+\sum_{k=0}^{N-1}P_{k}\left(2\omega_{\mathrm{c}}\omega_{k}+\omega_{k}^{2}\right)\right]
\end{aligned}
\end{equation}
Therefore, ${\left[{{\mathbf{F}}_{\mathbf{\theta}}}\right]_{1,1}}\propto{f_{c}}$.
EFI can be expressed as a function of M and $\Delta f$ as follows

\begin{equation}
\begin{aligned}\left[\mathbf{F}_{\theta}\right]_{1,1}= & \frac{2\alpha_{1}^{2}}{N_{0}}\left[2E_{\mathrm{T}}\omega_{\mathrm{c}}^{2}+\frac{E_{\mathrm{T}}\Delta\omega^{2}}{6}\left(4M^{2}-1\right)\right.\\
- & \left.\frac{E_{\mathrm{T}}\omega_{\mathrm{c}}^{2}}{2M^{2}}\frac{\sin^{2}(M\Delta\omega\tau)}{\sin^{2}(\Delta\omega\tau/2)}\right]
\end{aligned}
\end{equation}
where ${\omega_{\text{c}}}=2\pi{f_{\text{c}}}$, ${\omega_{k}}=2\pi{f_{k}}$
and $\Delta\omega=2\pi\Delta f$. Thus,the EFI for ranging can be
increased by using larger subcarrier spacing.According to the definition
of CRLB, as shown in Eq. (43), this shows that the larger the subcarrier
spacing is, the smaller the CRLB will be.

\section{Simulation results}

In this section, we evaluate the effects of signal and channel parameters
on the CRLB for OFDM ranging accuracy. We consider an OFDM system
with 121 subcarriers. The central frequency is 5 GHz and subcarrier
spacing of 4 MHz. One of the parameters can be changed in the experiment.
Four base stations are set on the same horizontal plane at a height
of 2m with coordinates of (0,0,2) (0,20,2) (20,0,2) (20,20,2). 100
Monte Carlo experiments were carried out in 100 randomly selected
locations on the base station plane to calculate MSE and RMSE.

\begin{figure}[htbp]
\begin{centering}
\textsf{\includegraphics[width=3.5in]{CRB_F_DELTA}} 
\par\end{centering}
\caption{CRLB baseline and MSE under different $\Delta f$}
\end{figure}

As shown in Fig. 8, the larger the frequency spacing used for positioning,
the smaller the CRLB, and the higher the ranging accuracy. It shows
that the proposed method has better performance in multipath environment.
Therefore, the frequency difference of the three subcarriers should
be chosen as large as possible. Select the 1st, 61st and 121st of
121 subcarriers. At this time, the subcarrier spacing is 240MHz, and
the corresponding MSE is 0.0017. The results of using the positioning
algorithm are slightly worse than CRLB, but remain basically the same.

\begin{figure}[htbp]
\begin{centering}
\textsf{\includegraphics[width=3.5in]{CRB_FC}} 
\par\end{centering}
\caption{CRLB baseline and MSE under different ${f_{c}}$}
\end{figure}

As shown in the figure above, the larger the fc, the Smaller the CRLB
of the ranging information, and the Fisher information is proportional
to the square of the $f_{c}$ . The MSE is 0.0096 at 3GHz and 0.0012
at 6GHz. The results show that in a 4G to 5G system, because the $f_{c}$
becomes larger, the ranging accuracy will perform better. Therefore,
the proposed method has better performance in high frequency signals.

\begin{figure}[htbp]
\begin{centering}
\textsf{\includegraphics[width=3.5in]{CRB_n}} 
\par\end{centering}
\caption{MSE of different methods when the number of subcarriers changes}
\end{figure}

As shown in Fig.10, the larger N is, the smaller the CRLB is, that
is, the more accurate the ranging information is. In accordance with
the previous section, Fisher information is proportional to the square
of N. However, a larger N means that the positioning signal needs
to occupy more subcarriers. At this time, the number of subcarriers
occupied by the communication data becomes smaller, that is, the communication
capacity is reduced. When single subcarrier is used, MSE is 0.0096
and when triple frequency is used, MSE is 0.0017. Therefore, when
three subcarriers are selected for positioning, MSE is increased by
80\%, and at the same time, it does not occupy too much communication
resources, thus ensuring that the original communication capacity
is not affected too much.

\begin{figure}[htbp]
\begin{centering}
\textsf{\includegraphics[width=3.5in]{CRB_snr}} 
\par\end{centering}
\caption{RMSE of different positioning methods}
\end{figure}

As shown in Fig.11. The RMSEs of ML algorithms of various indoor positioning
technologies are compared. These technologies are based on ranging
positioning methods. When using the same ML positioning algorithm,
ranging accuracy is positively related to positioning. Under the condition
of high signal-to-noise ratio, the positioning accuracy ratio of RFID
decreases rapidly. The multi-frequency positioning accuracy proposed
in this paper is much better than indoor positioning methods such
as PN, UWB and RFID.

\begin{table}[h]
\centering %表格居中
\caption{MSE of different menthods}
%表格标题
\begin{tabular}{|c|c|c|c|c|}
\hline 
SNR(dB)  & 5  & 10  & 15  & 20\tabularnewline
\hline 
proposed menthod  & 0.12042  & 0.08594  & 0.06133  & 0.04177\tabularnewline
\hline 
UWB  & 0.51036  & 0.32630  & 0.20862  & 0.13338\tabularnewline
\hline 
RFID  & 0.90820  & 0.42512  & 0.19899  & 0.09315\tabularnewline
\hline 
PN  & 2.12017  & 1.16961  & 0.68399  & 0.40138\tabularnewline
\hline 
\end{tabular}
\end{table}

\begin{figure}[htbp]
\begin{centering}
\textsf{\includegraphics[width=3.5in]{HDOP}} 
\par\end{centering}
\caption{HDOP of OFDM position system}
\end{figure}

Figure 12 shows the horizontal dilution of precision (HDOP). Four
base stations are set on the same horizontal plane at a height of
2m with coordinates of (0,0,2) (0,20,2) (20,0,2) (20,20,2). Where
coordinates are in meters. The positioning effect in the middle position
is worse than that in the peripheral position.Since the receiver has
similar angles to the four base stations at the center point, HDOP
is large and the calculation error is large. Therefore, although OFDM
can use multiple antennas of one base station to locate, its accuracy
is not as good as that of four base stations

\section{Conclusions}

This paper introduces a method for multi-frequency ranging using sub-carriers
in an OFDM communication system. The integrated OFDM communication
and positioning system is first introduced, including ranging subcarrier
selection and ranging frame format. Because the ranging positioning
only needs to know the position coordinates of the base station, the
base station coordinates of the transmitted signal are included in
the ranging frame. Subsequently, a ranging signal model transmitted
by the base station was established, and the CRLB of the ranging was
proved. Finally, simulation experiments prove that the RMSE of the
ranging and positioning system has superior accuracy compared with
other indoor positioning methods.Under experimental conditions, RMSE
can be less than 5cm.

\section*{References}

 \bibliographystyle{elsarticle-num}
\bibliography{OfdmConference}

\end{document}
